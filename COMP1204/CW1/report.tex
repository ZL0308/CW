% ----
% COMP1204 CW1 Report Document
% ----
\documentclass[]{article}
\usepackage{listings}
\usepackage{xcolor}
\usepackage{graphicx}
\usepackage[font=small,labelfont=bf]{caption}

% Set up the format of the code part
\definecolor{codegreen}{rgb}{0,0.6,0}
\definecolor{codegray}{rgb}{0.5,0.5,0.5}
\definecolor{codepurple}{rgb}{0.58,0,0.82}
\definecolor{backcolour}{rgb}{0.95,0.95,0.92}

% Simulate the same format of the bash.
\lstdefinestyle{codestyle}{
    backgroundcolor=\color{backcolour},   
    commentstyle=\color{codegreen},
    keywordstyle=\color{magenta},
    numberstyle=\tiny\color{codegray},
    stringstyle=\color{codepurple},
    basicstyle=\ttfamily\small,
    breakatwhitespace=false,         
    breaklines=true,                 
    captionpos=b,                    
    keepspaces=true,                 
    numbers=left,                    
    numbersep=5pt,                  
    showspaces=false,                
    showstringspaces=false,
    showtabs=false,                  
    tabsize=2
}

% Reduce the margin size, as they're quite big by default
\usepackage[margin=1in]{geometry}

\title{COMP1204: Data Management \\ Coursework One: Hurricane Monitoring }
% Update these!
\author{Zhengbin Lu \\ zl15g22}

% Actually start the report content
\begin{document}

% Add title, author and date info
\maketitle

\section{Introduction}
This coursework covers three parts, Unix scripting for file processing, report writing using Latex, and use of Git for version control. 

\vspace{2 ex}
\noindent
In this work, we're required to process data from the hurricane as a KML file by using Unix script, extract the important information, reformat the data and convert it into a CSV file, resolve the Git conflict by using version control and writing the report in latex format.

\vspace{2 ex}
\noindent
This report details how the Unix script work and the plots of hurricane after processing the data. Meanwhile, it contains the usage of resolving conflict.

\section{Create CSV Script}
Here is the detail of the Unix script. It cleans the data from KML file, reformat it and finally outputs into a CSV file.

\begin{lstlisting}[language=bash, style=codestyle, caption = CSV Script]
#!/bin/bash

#fetching the arguement of the input and output file 
input=$1
output=$2

# Displays the file which being processed, and the output file name.
echo "Converting $input -> $output"

#Output the header line which contains the basic information
# ">" - to initialize the file
echo "Timestamp,Latitude,Longitude,MinSeaLevelPressure,MaxIntensity" > $output

#cat - displays the content of the file
#grep - fetches the line which contains the keyword "UTC,N,mb,knots"
#sed - deletes the lines which contain the useless information by the keywords 
#sed - deletes the irrevelent words and notations
#paste - merges every four lines into one line, each line is separated by a space
#awk - prints the column which contains the relevant information
#sed - replaces ";" by a space
#sed - adjusts the format and replaces the unnecessary space between the words
#sed - deletes the space which at the end of the line
#">>" - to append the csv file
cat $input | grep -E 'UTC|N|mb|knots' | sed -E '/<(\/)?dtg|AL|TS|name|Name|Num|<TR>|TD/d' | sed -E 's/(]]>)|<(\/)?(B|td|tr|table)>//g' | paste -d ' ' - - - - | awk '{print $1, $2, $3, $4, ",", $5, $6, $7, $8, ",",$9, $10, "," $14, $15}' | sed 's/;/ /g' | sed -E 's/\s*,\s*/,/g' | sed 's/[[:space:]]*$//'>> $output

#Output the finish information
echo "Done!"
\end{lstlisting}

\section{Storm Plots}
The plots of three hurricanes, al012020, al102020, and al112020 are as follow:

\vspace{9 ex}
\noindent
\includegraphics[width=0.9\linewidth]{storm_plotal012020.png}
\captionof{figure}{Storm plot for al012020.kml}

\vspace{3 ex}
\noindent
\includegraphics[width=0.8\linewidth]{storm_plotal102020.png}
\captionof{figure}{Storm plot for al102020.kml}

\vspace{3 ex}
\noindent
\includegraphics[width=0.8\linewidth]{storm_plotal112020.png}
\captionof{figure}{Storm plot for al112020.kml}

\section{Git usage}
\begin{lstlisting}[language=bash, style=codestyle, caption = Branch master]
import pandas as pd
import matplotlib.pyplot as plt
import os
import glob
import math
user_key = 997

def plot_all_csv_pressure():
    path = os.getcwd()
    csv_files = glob.glob(os.path.join(path, '*.csv'))
    
    for f in csv_files:
        storm = pd.read_csv(f)
        storm['Pressure'].plot()
        plt.show()
\end{lstlisting}

\begin{lstlisting}[language=bash, style=codestyle, caption = Branch python-addon]
import pandas as pd
import os
import glob
import matplotlib.pyplot as plt
user_key= 274

def plot_all_csv_pressure():
    path = os.getcwd()
    csv_files = glob.glob(os.path.join(path, '*.csv'))
    
    fr f in csv_files:
        storm = pd.read_csv(f)
        storm['Pressure'].plot()
        plt.show()
		
def plot_all_csv_intensity():
    path = os.getcwd()
    csv_files = glob.glob(os.path.join(path, '*.csv'))
    
    for f in csv_files:
        storm = pd.read_csv(f)
        storm['Intensity'].plot()
        plt.show()
\end{lstlisting}

After executing the command and creating the conflict, I tried to merge the branch by using the 'git merge' command, and there was an error of version conflict. I modified the file by using vim, removed the "user key = 997" from the master branch, and the wrong function which was used to compute the pressure in the branch python-addon. Then I added the "user key = 274" and the "intensity function" from the branch python-addon, reordered the rest of the codes, removed the redundant part of the code, and resolved the conflict. After that, I added the comment by command 'git commit -am ' and pushed it into the remote repository by the command 'git push'. 

\vspace{3 ex}
\noindent
The code after conflict resolving as follow:
\begin{lstlisting}[language=bash, style=codestyle, caption = Script after conflict resolving]
import pandas as pd
import matplotlib.pyplot as plt
import os
import glob
import math
user_key = 997
user_key= 274

def plot_all_csv_pressure():
    path = os.getcwd()
    csv_files = glob.glob(os.path.join(path, '*.csv'))
    
    for f in csv_files:
        storm = pd.read_csv(f)
        storm['Pressure'].plot()
        plt.show()
		
def plot_all_csv_intensity():
    path = os.getcwd()
    csv_files = glob.glob(os.path.join(path, '*.csv'))
    
    for f in csv_files:
        storm = pd.read_csv(f)
        storm['Intensity'].plot()
        plt.show()
\end{lstlisting}
\end{document}
